Some features under development for the next major release are listed here.

\begin{itemize}
\item
  A programmer will be able to ask the user to select a node or edge via a text query. The appropriate incantations will take the form
  \begin{itemize}
    \item \texttt{getNode(\emph{message})}, where \emph{message} is a prompt
      (string); a popup window asks the user for the id of a node; the
      function will return the actual node; there is an error dialog if
      no node with that id exists.
    \item \texttt{getEdge(\emph{message})} -- same as \texttt{getNode}
      except that the user is prompted for two node id's; an error dialog
      arises if either id is non-existent or the edge is non-existent.
  \end{itemize}

\item
  New types/classes \texttt{NodeSet} and \texttt{EdgeSet} will be added. Both
  will be concrete classes and will have conversion constructors from
  \texttt{NodeList} and \texttt{EdgeList}, respectively. Eventually
  lists, queues, stacks and priority queues of nodes and edges will all be
  concrete and all graph, node and edge methods that return containers will
  return one of these. For example, \texttt{getNodes()} will return a
  \texttt{NodeList} instead of a \texttt{List<Node>}.

  When combined with the previous enhancement we can add functions such as\\
  \hspace*{2em}\mbox{\texttt{getEdge(\emph{prompt}, \emph{set}, \emph{message})}},\\
  where \emph{prompt} is as before, \emph{set} is
  a set from which the edge must be selected (e.g., the set of edges incident
  on a node) and \emph{message} is an error message to be given if the
  edge exists but is not a member of the given set.
\end{itemize}
