\begin{figure}

{\small
\begin{verbatim}
// parent in the disjoint set forest
final Node [] parent = new Node[ graph.numberOfNodes() ];

// standard disjoint set utilities; not doing union by rank or path
// compression; efficiency is not an issue
function INIT_SET( Node x ) {
    parent[x.getId()] = x;
}
function LINK( Node x, Node y ) {
    parent[x.getId()] = y;
}
function Node FIND_SET( Node x ) {
    if (x != parent[x.getId()])
        parent[x.getId()] = FIND_SET(parent[x.getId()]);
    return parent[x.getId()];
}
function UNION( Node x, Node y ) {
	LINK( FIND_SET(x), FIND_SET(y) );
}

for_nodes(u) {
    INIT_SET(u);
}

List<Edge> edgeList = getEdges();
Collections.sort( edgeList );

int totalWeight = 0;
for ( Edge e: edgeList ) {
    beginStep();
    Node h = e.getSourceNode(); Node t = e.getDestNode();
    h.mark(); t.mark(); // for display purposes only
    endStep();

    // if the vertices aren't part of the same set
    if ( FIND_SET(h) != FIND_SET(t) ) {
        // add the edge to the MST and highlight it
        e.setSelected( true );
        UNION(h, t);
        totalWeight += e.getWeight();
        graph.writeMessage( "Weight so far is " + totalWeight );
    }
    else {
        graph.writeMessage( "Vertices are already in the same component." );
    }

    beginStep(); h.unMark(); t.unMark(); endStep();
}
graph.writeMessage( "MST has total weight " + totalWeight );
\end{verbatim}
} % small

\caption{The implementation of Kruskal's algorithm animation.}
\label{fig:kruskals_algorithm}
\end{figure}


\begin{figure}[p]

\begin{center}
\fbox{
\includegraphics[scale=0.55]{X_kruskal_1}
}

(a) An edge is added to the tree.

\fbox{
\includegraphics[scale=0.55]{X_kruskal_2}
}

(b) The current edge creates a cycle.

\end{center}

\caption{Two steps in Kruskal's algorithm. A message at the top left of the window describes the state of the algorithm.}

\label{fig:kruskal_pictures}
\end{figure}


Another simple algorithm implementation is that of Kruskal's algorithm
for finding a minimum spanning tree (or forest) in a graph.
Fig.~\ref{fig:kruskals_algorithm}
shows the implemented animation of Kruskal's algorithm.
Additional Galant features illustrated here are:
\begin{itemize}

\item
Use of the keyword \verb+function+ to declare a function: this avoids the
syntactic complications of Java method declarations. In the case of
\verb+FIND_SET+, for example, you would normally have to say\\
\verb+static Node FIND_SET( Node x )+\\
and would get error messages about non-static methods in a static context
if you omitted the keyword \verb+static+.

\item
No need to use the Java keyword \verb+void+ to designate a function with
no return type. This is implicit if the return type is omitted as with
\verb+INIT_SET+, \verb+LINK+ and \verb+UNION+.

\item Implicit use of weights to sort the edges: weights are created by the
  explorer when editing a graph. The \verb+sort+ functions translates
  automatically into the relevant Java incantation.

\item
The ability to write messages during execution, as accomplished by the
\verb+display+ calls.

\end{itemize}

Two steps in the execution of the animation are shown
in Fig.~\ref{fig:kruskal_pictures}.
The two endpoints of an edge are marked when that edge is considered
for inclusion in the spanning tree.
If the endpoints are not in the same tree, the edge is added and highlighted
and the cost of the current tree is displayed in the message.
Otherwise the message reports that the endpoints are already in the same tree. 
