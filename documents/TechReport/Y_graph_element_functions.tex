\begin{table}
  \small
  \centering
  \begin{tabular}{| m{0.4\textwidth} | m{0.55\textwidth} |}
    \hline
    \textsf{boolean set(\emph{element}, String key, $\langle$\emph{type}$\rangle$ value)}
    &
    sets an arbitrary attribute, \textsf{key}, of the element to have a value of a given type, where
    the type is one of \textsf{Integer}, \textsf{Double}, \textsf{Boolean}
    or \textsf{String};
    in the special case of \textsf{Boolean} the third argument may be omitted
    and defaults to \textsf{true};
    so \textsf{set(v,"attr")} is equivalent to \textsf{set(v,"attr",true)};
    returns \textsf{true} if the element already has a value for the given attribute,
    \textsf{false} otherwise
    \\ \hline
    \shortstack[l]{
    \textsf{$\langle$\emph{type}$\rangle$ get$\langle$\emph{type}$\rangle$(\emph{element}, String key)}\\
    \textsf{Boolean is(\emph{element}, String key)}
    }
    &
    returns the value associated with \textsf{key} or \textsf{null}
    if the graph has no value of the given type for \textsf{key}, i.e.,
    if no
    \textsf{set(String~key,~$\langle$\emph{type}$\rangle$~value)} has occurred;
    in the special case of a \textsf{Boolean} attribute, the second formulation
    may be used
    \\ \hline
  \end{tabular}

  \caption{Functions that query and manipulate attributes of individual
    nodes and edges.
    Here, \emph{element} refers to either a \textsf{Node} or an \textsf{Edge},
    both the type and the formal parameter.
  }
  \label{tab:graph_element_functions}
\end{table}
