\begin{table}
  \centering
  \fbox{
    \begin{minipage}{0.9\textwidth}
      In the table below \emph{EdgeCollection} refers to any preexisting
      data structure whose elements are edges while \emph{NodeCollection}
      refers to one whose elements are nodes. Each type of structure can be
      initialized as empty or as containing elements of a given
      collection. As with arrays -- see page~\pageref{page:arrays} -- the
      \Code{new} operator creates a new instance of a structure.
    \end{minipage}
  }

  \medskip
  \begin{tabular}{|l | l |}
    \hline
    data structure & initializers \\
    \hline
    \Code{EdgeList}
    &
    \shortstack[l]{\Code{new EdgeList()}\\
      \Code{new EdgeList(\emph{EdgeCollection})}
      }
    \\
    \hline
    \Code{NodeList}
    &
    \shortstack[l]{\Code{new NodeList()}\\
      \Code{new NodeList(\emph{NodeCollection})}
      }
    \\
    \hline
    \Code{EdgeSet}
    &
    \shortstack[l]{\Code{new EdgeSet()}\\
      \Code{new EdgeSet(\emph{EdgeCollection})}
      }
    \\
    \hline
    \Code{NodeSet}
    &
    \shortstack[l]{\Code{new NodeSet()}\\
      \Code{new NodeSet(\emph{NodeCollection})}
      }
    \\
    \hline
    \Code{EdgeQueue}
    &
    \shortstack[l]{\Code{new EdgeQueue()}\\
      \Code{new EdgeQueue(\emph{EdgeCollection})}
      }
    \\
    \hline
    \Code{NodeQueue}
    &
    \shortstack[l]{\Code{new NodeQueue()}\\
      \Code{new NodeQueue(\emph{NodeCollection})}
      }
    \\
    \hline
    \Code{EdgePriorityQueue}
    &
    \shortstack[l]{\Code{new EdgePriorityQueue()}\\
      \Code{new EdgePriorityQueue(\emph{EdgeCollection})}
      }
    \\
    \hline
    \Code{NodePriorityQueue}
    &
    \shortstack[l]{\Code{new NodePriorityQueue()}\\
      \Code{new NodePriorityQueue(\emph{NodeCollection})}
      }
    \\
    \hline
  \end{tabular}

  \caption{Basic Galant data structures and initialization.}
  \label{tab:data_structure_summary}
\end{table}

% [Last modified: 2017 01 17 at 19:59:54 GMT]
